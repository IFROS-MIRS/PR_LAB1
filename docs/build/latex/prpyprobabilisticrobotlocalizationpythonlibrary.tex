%% Generated by Sphinx.
\def\sphinxdocclass{report}
\documentclass[letterpaper,10pt,english]{sphinxmanual}
\ifdefined\pdfpxdimen
   \let\sphinxpxdimen\pdfpxdimen\else\newdimen\sphinxpxdimen
\fi \sphinxpxdimen=.75bp\relax
\ifdefined\pdfimageresolution
    \pdfimageresolution= \numexpr \dimexpr1in\relax/\sphinxpxdimen\relax
\fi
%% let collapsible pdf bookmarks panel have high depth per default
\PassOptionsToPackage{bookmarksdepth=5}{hyperref}

\PassOptionsToPackage{booktabs}{sphinx}
\PassOptionsToPackage{colorrows}{sphinx}

\PassOptionsToPackage{warn}{textcomp}
\usepackage[utf8]{inputenc}
\ifdefined\DeclareUnicodeCharacter
% support both utf8 and utf8x syntaxes
  \ifdefined\DeclareUnicodeCharacterAsOptional
    \def\sphinxDUC#1{\DeclareUnicodeCharacter{"#1}}
  \else
    \let\sphinxDUC\DeclareUnicodeCharacter
  \fi
  \sphinxDUC{00A0}{\nobreakspace}
  \sphinxDUC{2500}{\sphinxunichar{2500}}
  \sphinxDUC{2502}{\sphinxunichar{2502}}
  \sphinxDUC{2514}{\sphinxunichar{2514}}
  \sphinxDUC{251C}{\sphinxunichar{251C}}
  \sphinxDUC{2572}{\textbackslash}
\fi
\usepackage{cmap}
\usepackage[T1]{fontenc}
\usepackage{amsmath,amssymb,amstext}
\usepackage{babel}



\usepackage{tgtermes}
\usepackage{tgheros}
\renewcommand{\ttdefault}{txtt}



\usepackage[Bjarne]{fncychap}
\usepackage{sphinx}

\fvset{fontsize=auto}
\usepackage{geometry}


% Include hyperref last.
\usepackage{hyperref}
% Fix anchor placement for figures with captions.
\usepackage{hypcap}% it must be loaded after hyperref.
% Set up styles of URL: it should be placed after hyperref.
\urlstyle{same}


\usepackage{sphinxmessages}
\setcounter{tocdepth}{1}



\title{prpy: Probabilistic Robot Localization Python Library}
\date{Oct 02, 2023}
\release{0.1}
\author{Pere Ridao}
\newcommand{\sphinxlogo}{\vbox{}}
\renewcommand{\releasename}{Release}
\makeindex
\begin{document}

\ifdefined\shorthandoff
  \ifnum\catcode`\=\string=\active\shorthandoff{=}\fi
  \ifnum\catcode`\"=\active\shorthandoff{"}\fi
\fi

\pagestyle{empty}
\sphinxmaketitle
\pagestyle{plain}
\sphinxtableofcontents
\pagestyle{normal}
\phantomsection\label{\detokenize{index::doc}}


\sphinxAtStartPar
\sphinxstylestrong{Probabilistic Robot Localization} is Python Library containing the main algorithms explained in the \sphinxstylestrong{Probabilisitic Robot Localization} Book used in the \sphinxstylestrong{Probabilisitic Robotics} and the \sphinxstylestrong{Hands\sphinxhyphen{}on Localization} Courses of the \sphinxstylestrong{Intelligent Field Robotic Systems (IFRoS)} European Erasmus Mundus Master.

\begin{sphinxadmonition}{note}{Note:}
\sphinxAtStartPar
This documentation is still under construction.
\end{sphinxadmonition}


\chapter{API:}
\label{\detokenize{index:api}}
\sphinxstepscope


\section{Pose Representation}
\label{\detokenize{compounding:pose-representation}}\label{\detokenize{compounding::doc}}

\subsection{Pose 3DOF}
\label{\detokenize{compounding:pose-3dof}}
\begin{figure}[htbp]
\centering

\noindent\sphinxincludegraphics[scale=0.75]{{Pose3D}.png}
\end{figure}
\index{Pose3D (class in Pose3D)@\spxentry{Pose3D}\spxextra{class in Pose3D}}

\begin{fulllineitems}
\phantomsection\label{\detokenize{compounding:Pose3D.Pose3D}}
\pysigstartsignatures
\pysiglinewithargsret{\sphinxbfcode{\sphinxupquote{class\DUrole{w}{ }}}\sphinxcode{\sphinxupquote{Pose3D.}}\sphinxbfcode{\sphinxupquote{Pose3D}}}{\sphinxparam{\DUrole{n}{input\_array}}}{}
\pysigstopsignatures
\sphinxAtStartPar
Bases: \sphinxcode{\sphinxupquote{ndarray}}

\sphinxAtStartPar
Definition of a robot pose in 3 DOF (x, y, yaw). The class inherits from a ndarray.
This class extends the ndarray with the \$oplus\$ and \$ominus\$ operators and the corresponding Jacobians.
\index{oplus() (Pose3D.Pose3D method)@\spxentry{oplus()}\spxextra{Pose3D.Pose3D method}}

\begin{fulllineitems}
\phantomsection\label{\detokenize{compounding:Pose3D.Pose3D.oplus}}
\pysigstartsignatures
\pysiglinewithargsret{\sphinxbfcode{\sphinxupquote{oplus}}}{\sphinxparam{\DUrole{n}{BxC}}}{}
\pysigstopsignatures
\sphinxAtStartPar
Given a Pose3D object \sphinxstyleemphasis{AxB} (the self object) and a Pose3D object \sphinxstyleemphasis{BxC}, it returns the Pose3D object \sphinxstyleemphasis{AxC}.
\begin{equation*}
\begin{split}\mathbf{{^A}x_B} &= \begin{bmatrix} ^Ax_B & ^Ay_B & ^A\psi_B \end{bmatrix}^T \\
\mathbf{{^B}x_C} &= \begin{bmatrix} ^Bx_C & ^By_C & & ^B\psi_C \end{bmatrix}^T \\\end{split}
\end{equation*}
\sphinxAtStartPar
The operation is defined as:
\begin{equation}\label{equation:compounding:eq-oplus3dof}
\begin{split}\mathbf{{^A}x_C} &= \mathbf{{^A}x_B} \oplus \mathbf{{^B}x_C} =
\begin{bmatrix}
    ^Ax_B + ^Bx_C  \cos(^A\psi_B) - ^By_C  \sin(^A\psi_B) \\
    ^Ay_B + ^Bx_C  \sin(^A\psi_B) + ^By_C  \cos(^A\psi_B) \\
    ^A\psi_B + ^B\psi_C
\end{bmatrix}\end{split}
\end{equation}\begin{quote}\begin{description}
\sphinxlineitem{Parameters}
\sphinxAtStartPar
\sphinxstyleliteralstrong{\sphinxupquote{BxC}} \textendash{} C\sphinxhyphen{}Frame pose expressed in B\sphinxhyphen{}Frame coordinates

\sphinxlineitem{Returns}
\sphinxAtStartPar
C\sphinxhyphen{}Frame pose expressed in A\sphinxhyphen{}Frame coordinates

\end{description}\end{quote}

\end{fulllineitems}

\index{J\_1oplus() (Pose3D.Pose3D method)@\spxentry{J\_1oplus()}\spxextra{Pose3D.Pose3D method}}

\begin{fulllineitems}
\phantomsection\label{\detokenize{compounding:Pose3D.Pose3D.J_1oplus}}
\pysigstartsignatures
\pysiglinewithargsret{\sphinxbfcode{\sphinxupquote{J\_1oplus}}}{\sphinxparam{\DUrole{n}{BxC}}}{}
\pysigstopsignatures
\sphinxAtStartPar
Jacobian of the pose compounding operation (eq. \eqref{equation:compounding:eq-oplus3dof}) with respect to the first pose:
\begin{equation}\label{equation:compounding:eq-J1oplus3dof}
\begin{split}J_{1\oplus}=\frac{\partial  ^Ax_B \oplus ^Bx_C}{\partial ^Ax_B} =
\begin{bmatrix}
    1 & 0 &  -^Bx_C \sin(^A\psi_B) - ^By_C \cos(^A\psi_B) \\
    0 & 1 &  ^Bx_C \cos(^A\psi_B) - ^By_C \sin(^A\psi_B) \\
    0 & 0 & 1
\end{bmatrix}\end{split}
\end{equation}
\sphinxAtStartPar
The method returns a numerical matrix containing the evaluation of the Jacobian for the pose \sphinxstyleemphasis{AxB} (the self object) and the \$2\textasciicircum{}\{nd\}\$ posepose \sphinxstyleemphasis{BxC}.
\begin{quote}\begin{description}
\sphinxlineitem{Parameters}
\sphinxAtStartPar
\sphinxstyleliteralstrong{\sphinxupquote{BxC}} \textendash{} 2nd pose

\sphinxlineitem{Returns}
\sphinxAtStartPar
Evaluation of the \(J_{1\oplus}\) Jacobian of the pose compounding operation with respect to the first pose (eq. \eqref{equation:compounding:eq-J1oplus3dof})

\end{description}\end{quote}

\end{fulllineitems}

\index{J\_2oplus() (Pose3D.Pose3D method)@\spxentry{J\_2oplus()}\spxextra{Pose3D.Pose3D method}}

\begin{fulllineitems}
\phantomsection\label{\detokenize{compounding:Pose3D.Pose3D.J_2oplus}}
\pysigstartsignatures
\pysiglinewithargsret{\sphinxbfcode{\sphinxupquote{J\_2oplus}}}{}{}
\pysigstopsignatures
\sphinxAtStartPar
Jacobian of the pose compounding operation (\eqref{equation:compounding:eq-oplus3dof}) with respect to the second pose:
\begin{equation}\label{equation:compounding:eq-J2oplus3dof}
\begin{split}J_{2\oplus}=\frac{\partial  ^Ax_B \oplus ^Bx_C}{\partial ^Bx_C} =
\begin{bmatrix}
    \cos(^A\psi_B) & -\sin(^A\psi_B) & 0  \\
    \sin(^A\psi_B) & \cos(^A\psi_B) & 0  \\
    0 & 0 & 1
\end{bmatrix}\end{split}
\end{equation}
\sphinxAtStartPar
The method returns a numerical matrix containing the evaluation of the Jacobian for the \$1\textasciicircum{}\{st\}\$ posepose \sphinxstyleemphasis{AxB} (the self object).
\begin{quote}\begin{description}
\sphinxlineitem{Returns}
\sphinxAtStartPar
Evaluation of the \(J_{2\oplus}\) Jacobian of the pose compounding operation with respect to the second pose (eq. \eqref{equation:compounding:eq-J2oplus3dof})

\end{description}\end{quote}

\end{fulllineitems}

\index{ominus() (Pose3D.Pose3D method)@\spxentry{ominus()}\spxextra{Pose3D.Pose3D method}}

\begin{fulllineitems}
\phantomsection\label{\detokenize{compounding:Pose3D.Pose3D.ominus}}
\pysigstartsignatures
\pysiglinewithargsret{\sphinxbfcode{\sphinxupquote{ominus}}}{}{}
\pysigstopsignatures
\sphinxAtStartPar
Inverse pose compounding of the \sphinxstyleemphasis{AxB} pose (the self objetc):
\begin{equation}\label{equation:compounding:eq-ominus3dof}
\begin{split}^Bx_A = \ominus ^Ax_B =
\begin{bmatrix}
    -^Ax_B \cos(^A\psi_B) - ^Ay_B \sin(^A\psi_B) \\
    ^Ax_B \sin(^A\psi_B) - ^Ay_B \cos(^A\psi_B) \\
    -^A\psi_B
\end{bmatrix}\end{split}
\end{equation}\begin{quote}\begin{description}
\sphinxlineitem{Returns}
\sphinxAtStartPar
A\sphinxhyphen{}Frame pose expressed in B\sphinxhyphen{}Frame coordinates (eq. \eqref{equation:compounding:eq-ominus3dof})

\end{description}\end{quote}

\end{fulllineitems}

\index{J\_ominus() (Pose3D.Pose3D method)@\spxentry{J\_ominus()}\spxextra{Pose3D.Pose3D method}}

\begin{fulllineitems}
\phantomsection\label{\detokenize{compounding:Pose3D.Pose3D.J_ominus}}
\pysigstartsignatures
\pysiglinewithargsret{\sphinxbfcode{\sphinxupquote{J\_ominus}}}{}{}
\pysigstopsignatures
\sphinxAtStartPar
Jacobian of the inverse pose compounding operation (\eqref{equation:compounding:eq-oplus3dof}) with respect the pose \sphinxstyleemphasis{AxB} (the self object):
\begin{equation}\label{equation:compounding:eq-Jominus3dof}
\begin{split}J_{\ominus}=\frac{\partial  \ominus ^Ax_B}{\partial ^Ax_B} =
\begin{bmatrix}
    -\cos(^A\psi_B) & -\sin(^A\psi_B) &  ^Ax_B \sin(^A\psi_B) - ^Ay_B \cos(^A\psi_B) \\
    \sin(^A\psi_B) & -\cos(^A\psi_B) &  ^Ax_B \cos(^A\psi_B) + ^Ay_B \sin(^A\psi_B) \\
    0 & 0 & -1
\end{bmatrix}\end{split}
\end{equation}
\sphinxAtStartPar
Returns the numerical matrix containing the evaluation of the Jacobian for the pose \sphinxstyleemphasis{AxB} (the self object).
\begin{quote}\begin{description}
\sphinxlineitem{Returns}
\sphinxAtStartPar
Evaluation of the \(J_{\ominus}\) Jacobian of the inverse pose compounding operation with respect to the pose (eq. \eqref{equation:compounding:eq-Jominus3dof})

\end{description}\end{quote}

\end{fulllineitems}


\end{fulllineitems}


\sphinxstepscope


\section{Robot Simulation}
\label{\detokenize{robot_simulation:robot-simulation}}\label{\detokenize{robot_simulation::doc}}
\begin{figure}[htbp]
\centering
\capstart

\noindent\sphinxincludegraphics[scale=0.75]{{SimulatedRobot}.png}
\caption{SimulatedRobot Class Diagram.}\label{\detokenize{robot_simulation:id1}}\end{figure}
\index{SimulatedRobot (class in SimulatedRobot)@\spxentry{SimulatedRobot}\spxextra{class in SimulatedRobot}}

\begin{fulllineitems}
\phantomsection\label{\detokenize{robot_simulation:SimulatedRobot.SimulatedRobot}}
\pysigstartsignatures
\pysiglinewithargsret{\sphinxbfcode{\sphinxupquote{class\DUrole{w}{ }}}\sphinxcode{\sphinxupquote{SimulatedRobot.}}\sphinxbfcode{\sphinxupquote{SimulatedRobot}}}{\sphinxparam{\DUrole{n}{xs0}}\sphinxparamcomma \sphinxparam{\DUrole{n}{map}\DUrole{o}{=}\DUrole{default_value}{{[}{]}}}\sphinxparamcomma \sphinxparam{\DUrole{o}{*}\DUrole{n}{args}}}{}
\pysigstopsignatures
\sphinxAtStartPar
Bases: \sphinxcode{\sphinxupquote{object}}

\sphinxAtStartPar
This is the base class to simulate a robot. There are two operative frames: the world  N\sphinxhyphen{}Frame (North East Down oriented) and the robot body frame body B\sphinxhyphen{}Frame.
Each robot has a motion model and a measurement model. The motion model is used to simulate the robot motion and the measurement model is used to simulate the robot measurements.

\sphinxAtStartPar
\sphinxstylestrong{All Robot simulation classes must derive from this class} .
\index{dt (SimulatedRobot.SimulatedRobot attribute)@\spxentry{dt}\spxextra{SimulatedRobot.SimulatedRobot attribute}}

\begin{fulllineitems}
\phantomsection\label{\detokenize{robot_simulation:SimulatedRobot.SimulatedRobot.dt}}
\pysigstartsignatures
\pysigline{\sphinxbfcode{\sphinxupquote{dt}}\sphinxbfcode{\sphinxupquote{\DUrole{w}{ }\DUrole{p}{=}\DUrole{w}{ }0.1}}}
\pysigstopsignatures
\sphinxAtStartPar
class attribute containing sample time of the simulation

\end{fulllineitems}

\index{\_\_init\_\_() (SimulatedRobot.SimulatedRobot method)@\spxentry{\_\_init\_\_()}\spxextra{SimulatedRobot.SimulatedRobot method}}

\begin{fulllineitems}
\phantomsection\label{\detokenize{robot_simulation:SimulatedRobot.SimulatedRobot.__init__}}
\pysigstartsignatures
\pysiglinewithargsret{\sphinxbfcode{\sphinxupquote{\_\_init\_\_}}}{\sphinxparam{\DUrole{n}{xs0}}\sphinxparamcomma \sphinxparam{\DUrole{n}{map}\DUrole{o}{=}\DUrole{default_value}{{[}{]}}}\sphinxparamcomma \sphinxparam{\DUrole{o}{*}\DUrole{n}{args}}}{}
\pysigstopsignatures\begin{quote}\begin{description}
\sphinxlineitem{Parameters}\begin{itemize}
\item {} 
\sphinxAtStartPar
\sphinxstyleliteralstrong{\sphinxupquote{xs0}} \textendash{} initial simulated robot state \(x_{s_0}\) used to initialize the the motion model

\item {} 
\sphinxAtStartPar
\sphinxstyleliteralstrong{\sphinxupquote{map}} \textendash{} feature map of the environment \(M=[^Nx_{F_1}^T,...,^Nx_{F_{nf}}^T]^T\)

\end{itemize}

\end{description}\end{quote}

\sphinxAtStartPar
Constructor. First, it initializes the robot simulation defining the following attributes:
\begin{itemize}
\item {} 
\sphinxAtStartPar
\sphinxstylestrong{k} : time step

\item {} 
\sphinxAtStartPar
\sphinxstylestrong{Qsk} : \sphinxstylestrong{To be defined in the derived classes}. Object attribute containing Covariance of the simulation motion model noise

\item {} 
\sphinxAtStartPar
\sphinxstylestrong{usk} : \sphinxstylestrong{To be defined in the derived classes}. Object attribute contining the simulated input to the motion model

\item {} 
\sphinxAtStartPar
\sphinxstylestrong{xsk} : \sphinxstylestrong{To be defined in the derived classes}. Object attribute contining the current simulated robot state

\item {} 
\sphinxAtStartPar
\sphinxstylestrong{zsk} : \sphinxstylestrong{To be defined in the derived classes}. Object attribute contining the current simulated robot measurement

\item {} 
\sphinxAtStartPar
\sphinxstylestrong{Rsk} : \sphinxstylestrong{To be defined in the derived classes}. Object attribute contining the observation noise covariance matrix

\item {} 
\sphinxAtStartPar
\sphinxstylestrong{xsk} : current pose is the initial state

\item {} 
\sphinxAtStartPar
\sphinxstylestrong{xsk\_1} : previouse state is the initial robot state

\item {} 
\sphinxAtStartPar
\sphinxstylestrong{M} : position of the features in the N\sphinxhyphen{}Frame

\item {} 
\sphinxAtStartPar
\sphinxstylestrong{nf} : number of features

\end{itemize}

\sphinxAtStartPar
Then, the robot animation is initialized defining the following attributes:
\begin{itemize}
\item {} 
\sphinxAtStartPar
\sphinxstylestrong{vehicleIcon} : Path file of the image of the robot to be used in the animation

\item {} 
\sphinxAtStartPar
\sphinxstylestrong{vehicleFig} : Figure of the robot to be used in the animation

\item {} 
\sphinxAtStartPar
\sphinxstylestrong{vehicleAxes} : Axes of the robot to be used in the animation

\item {} 
\sphinxAtStartPar
\sphinxstylestrong{xTraj} : list containing the x coordinates of the robot trajectory

\item {} 
\sphinxAtStartPar
\sphinxstylestrong{yTraj} : list containing the y coordinates of the robot trajectory

\item {} 
\sphinxAtStartPar
\sphinxstylestrong{visualizationInterval} : time\sphinxhyphen{}steps interval between two consecutive frames of the animation

\end{itemize}

\end{fulllineitems}

\index{PlotRobot() (SimulatedRobot.SimulatedRobot method)@\spxentry{PlotRobot()}\spxextra{SimulatedRobot.SimulatedRobot method}}

\begin{fulllineitems}
\phantomsection\label{\detokenize{robot_simulation:SimulatedRobot.SimulatedRobot.PlotRobot}}
\pysigstartsignatures
\pysiglinewithargsret{\sphinxbfcode{\sphinxupquote{PlotRobot}}}{}{}
\pysigstopsignatures
\sphinxAtStartPar
Updates the plot of the robot at the current pose

\end{fulllineitems}

\index{fs() (SimulatedRobot.SimulatedRobot method)@\spxentry{fs()}\spxextra{SimulatedRobot.SimulatedRobot method}}

\begin{fulllineitems}
\phantomsection\label{\detokenize{robot_simulation:SimulatedRobot.SimulatedRobot.fs}}
\pysigstartsignatures
\pysiglinewithargsret{\sphinxbfcode{\sphinxupquote{fs}}}{\sphinxparam{\DUrole{n}{xsk\_1}}\sphinxparamcomma \sphinxparam{\DUrole{n}{usk}}}{}
\pysigstopsignatures
\sphinxAtStartPar
Motion model used to simulate the robot motion. Computes the current robot state \(x_k\) given the previous robot state \(x_{k-1}\) and the input \(u_k\).
It also updates the object attributes \(xsk\), \(xsk_1\) and  \(usk\) to be made them available for plotting purposes.
\sphinxstyleemphasis{To be overriden in child class}.
\begin{quote}\begin{description}
\sphinxlineitem{Parameters}\begin{itemize}
\item {} 
\sphinxAtStartPar
\sphinxstyleliteralstrong{\sphinxupquote{xsk\_1}} \textendash{} previous robot state \(x_{k-1}\)

\item {} 
\sphinxAtStartPar
\sphinxstyleliteralstrong{\sphinxupquote{usk}} \textendash{} model input \(u_{s_k}\)

\end{itemize}

\sphinxlineitem{Returns}
\sphinxAtStartPar
current robot state \(x_k\)

\end{description}\end{quote}

\end{fulllineitems}

\index{SetMap() (SimulatedRobot.SimulatedRobot method)@\spxentry{SetMap()}\spxextra{SimulatedRobot.SimulatedRobot method}}

\begin{fulllineitems}
\phantomsection\label{\detokenize{robot_simulation:SimulatedRobot.SimulatedRobot.SetMap}}
\pysigstartsignatures
\pysiglinewithargsret{\sphinxbfcode{\sphinxupquote{SetMap}}}{\sphinxparam{\DUrole{n}{map}}}{}
\pysigstopsignatures
\sphinxAtStartPar
Initializes the map of the environment.

\end{fulllineitems}

\index{\_PlotSample() (SimulatedRobot.SimulatedRobot method)@\spxentry{\_PlotSample()}\spxextra{SimulatedRobot.SimulatedRobot method}}

\begin{fulllineitems}
\phantomsection\label{\detokenize{robot_simulation:SimulatedRobot.SimulatedRobot._PlotSample}}
\pysigstartsignatures
\pysiglinewithargsret{\sphinxbfcode{\sphinxupquote{\_PlotSample}}}{\sphinxparam{\DUrole{n}{x}}\sphinxparamcomma \sphinxparam{\DUrole{n}{P}}\sphinxparamcomma \sphinxparam{\DUrole{n}{n}}}{}
\pysigstopsignatures
\sphinxAtStartPar
Plots n samples of a multivariate gaussian distribution. This function is used only for testing, to plot the
uncertainty through samples.
:param x: mean pose of the distribution
:param P: covariance of the distribution
:param n: number of samples to plot

\end{fulllineitems}


\end{fulllineitems}



\subsection{3 DOF Diferential Drive Robot Simulation}
\label{\detokenize{robot_simulation:dof-diferential-drive-robot-simulation}}
\begin{figure}[htbp]
\centering
\capstart

\noindent\sphinxincludegraphics[scale=0.75]{{DifferentialDriveSimulatedRobot}.png}
\caption{DifferentialDriveSimulatedRobot Class Diagram.}\label{\detokenize{robot_simulation:id2}}\end{figure}
\index{DifferentialDriveSimulatedRobot (class in DifferentialDriveSimulatedRobot)@\spxentry{DifferentialDriveSimulatedRobot}\spxextra{class in DifferentialDriveSimulatedRobot}}

\begin{fulllineitems}
\phantomsection\label{\detokenize{robot_simulation:DifferentialDriveSimulatedRobot.DifferentialDriveSimulatedRobot}}
\pysigstartsignatures
\pysiglinewithargsret{\sphinxbfcode{\sphinxupquote{class\DUrole{w}{ }}}\sphinxcode{\sphinxupquote{DifferentialDriveSimulatedRobot.}}\sphinxbfcode{\sphinxupquote{DifferentialDriveSimulatedRobot}}}{\sphinxparam{\DUrole{n}{xs0}}\sphinxparamcomma \sphinxparam{\DUrole{n}{map}\DUrole{o}{=}\DUrole{default_value}{{[}{]}}}\sphinxparamcomma \sphinxparam{\DUrole{o}{*}\DUrole{n}{args}}}{}
\pysigstopsignatures
\sphinxAtStartPar
Bases: {\hyperref[\detokenize{robot_simulation:SimulatedRobot.SimulatedRobot}]{\sphinxcrossref{\sphinxcode{\sphinxupquote{SimulatedRobot}}}}}

\sphinxAtStartPar
This class implements a simulated differential drive robot. It inherits from the \sphinxcode{\sphinxupquote{SimulatedRobot}} class and
overrides some of its methods to define the differential drive robot motion model.
\index{\_\_init\_\_() (DifferentialDriveSimulatedRobot.DifferentialDriveSimulatedRobot method)@\spxentry{\_\_init\_\_()}\spxextra{DifferentialDriveSimulatedRobot.DifferentialDriveSimulatedRobot method}}

\begin{fulllineitems}
\phantomsection\label{\detokenize{robot_simulation:DifferentialDriveSimulatedRobot.DifferentialDriveSimulatedRobot.__init__}}
\pysigstartsignatures
\pysiglinewithargsret{\sphinxbfcode{\sphinxupquote{\_\_init\_\_}}}{\sphinxparam{\DUrole{n}{xs0}}\sphinxparamcomma \sphinxparam{\DUrole{n}{map}\DUrole{o}{=}\DUrole{default_value}{{[}{]}}}\sphinxparamcomma \sphinxparam{\DUrole{o}{*}\DUrole{n}{args}}}{}
\pysigstopsignatures\begin{quote}\begin{description}
\sphinxlineitem{Parameters}\begin{itemize}
\item {} 
\sphinxAtStartPar
\sphinxstyleliteralstrong{\sphinxupquote{xs0}} \textendash{} initial simulated robot state \(\mathbf{x_{s_0}}=[^Nx{_{s_0}}~^Ny{_{s_0}}~^N\psi{_{s_0}}~]^T\) used to initialize the  motion model

\item {} 
\sphinxAtStartPar
\sphinxstyleliteralstrong{\sphinxupquote{map}} \textendash{} feature map of the environment \(M=[^Nx_{F_1},...,^Nx_{F_{nf}}]\)

\end{itemize}

\end{description}\end{quote}

\sphinxAtStartPar
Initializes the simulated differential drive robot. Overrides some of the object attributes of the parent class \sphinxcode{\sphinxupquote{SimulatedRobot}} to define the differential drive robot motion model:
\begin{itemize}
\item {} 
\sphinxAtStartPar
\sphinxstylestrong{Qsk} : Object attribute containing Covariance of the simulation motion model noise.

\end{itemize}
\begin{equation}\label{equation:robot_simulation:eq:Qsk}
\begin{split}Q_k=\begin{bmatrix}\sigma_{\dot u}^2 & 0 & 0\\
0 & \sigma_{\dot v}^2 & 0 \\
0 & 0 & \sigma_{\dot r}^2 \\
\end{bmatrix}\end{split}
\end{equation}\begin{itemize}
\item {} 
\sphinxAtStartPar
\sphinxstylestrong{usk} : Object attribute containing the simulated input to the motion model containing the forward velocity \(u_k\) and the angular velocity \(r_k\)

\end{itemize}
\begin{equation}\label{equation:robot_simulation:eq:usk}
\begin{split}\bf{u_k}=\begin{bmatrix}u_k & r_k\end{bmatrix}^T\end{split}
\end{equation}\begin{itemize}
\item {} 
\sphinxAtStartPar
\sphinxstylestrong{xsk} : Object attribute containing the current simulated robot state

\end{itemize}
\begin{equation}\label{equation:robot_simulation:eq:xsk}
\begin{split}x_k=\begin{bmatrix}{^N}x_k & {^N}y_k & {^N}\theta_k & {^B}u_k & {^B}v_k & {^B}r_k\end{bmatrix}^T\end{split}
\end{equation}
\sphinxAtStartPar
where \({^N}x_k\), \({^N}y_k\) and \({^N}\theta_k\) are the robot position and orientation in the world N\sphinxhyphen{}Frame, and \({^B}u_k\), \({^B}v_k\) and \({^B}r_k\) are the robot linear and angular velocities in the robot B\sphinxhyphen{}Frame.
\begin{itemize}
\item {} 
\sphinxAtStartPar
\sphinxstylestrong{zsk} : Object attribute containing \(z_{s_k}=[n_L~n_R]^T\) observation vector containing number of pulses read from the left and right wheel encoders.

\item {} 
\sphinxAtStartPar
\sphinxstylestrong{Rsk} : Object attribute containing \(R_{s_k}=diag(\sigma_L^2,\sigma_R^2)\) covariance matrix of the noise of the read pulses\textasciigrave{}.

\item {} 
\sphinxAtStartPar
\sphinxstylestrong{wheelBase} : Object attribute containing the distance between the wheels of the robot (\(w=0.5\) m)

\item {} 
\sphinxAtStartPar
\sphinxstylestrong{wheelRadius} : Object attribute containing the radius of the wheels of the robot (\(R=0.1\) m)

\item {} 
\sphinxAtStartPar
\sphinxstylestrong{pulses\_x\_wheelTurn} : Object attribute containing the number of pulses per wheel turn (\(pulseXwheelTurn=1024\) pulses)

\item {} 
\sphinxAtStartPar
\sphinxstylestrong{Polar2D\_max\_range} : Object attribute containing the maximum Polar2D range (\(Polar2D_max_range=50\) m) at which the robot can detect features.

\item {} 
\sphinxAtStartPar
\sphinxstylestrong{Polar2D\_feature\_reading\_frequency} : Object attribute containing the frequency of Polar2D feature readings (50 tics \sphinxhyphen{}sample times\sphinxhyphen{})

\item {} 
\sphinxAtStartPar
\sphinxstylestrong{Rfp} : Object attribute containing the covariance of the simulated Polar2D feature noise (\(R_{fp}=diag(\sigma_{\rho}^2,\sigma_{\phi}^2)\))

\end{itemize}

\sphinxAtStartPar
Check the parent class \sphinxcode{\sphinxupquote{prpy.SimulatedRobot}} to know the rest of the object attributes.

\end{fulllineitems}

\index{fs() (DifferentialDriveSimulatedRobot.DifferentialDriveSimulatedRobot method)@\spxentry{fs()}\spxextra{DifferentialDriveSimulatedRobot.DifferentialDriveSimulatedRobot method}}

\begin{fulllineitems}
\phantomsection\label{\detokenize{robot_simulation:DifferentialDriveSimulatedRobot.DifferentialDriveSimulatedRobot.fs}}
\pysigstartsignatures
\pysiglinewithargsret{\sphinxbfcode{\sphinxupquote{fs}}}{\sphinxparam{\DUrole{n}{xsk\_1}}\sphinxparamcomma \sphinxparam{\DUrole{n}{usk}}}{}
\pysigstopsignatures
\sphinxAtStartPar
Motion model used to simulate the robot motion. Computes the current robot state \(x_k\) given the previous robot state \(x_{k-1}\) and the input \(u_k\):
\begin{equation}\label{equation:robot_simulation:eq:fs}
\begin{split}\eta_{s_{k-1}}&=\begin{bmatrix}x_{s_{k-1}} & y_{s_{k-1}} & \theta_{s_{k-1}}\end{bmatrix}^T\\
\nu_{s_{k-1}}&=\begin{bmatrix} u_{s_{k-1}} &  v_{s_{k-1}} & r_{s_{k-1}}\end{bmatrix}^T\\
x_{s_{k-1}}&=\begin{bmatrix}\eta_{s_{k-1}}^T & \nu_{s_{k-1}}^T\end{bmatrix}^T\\
u_{s_k}&=\nu_{d}=\begin{bmatrix} u_d& r_d\end{bmatrix}^T\\
w_{s_k}&=\dot \nu_{s_k}\\
x_{s_k}&=f_s(x_{s_{k-1}},u_{s_k},w_{s_k}) \\
&=\begin{bmatrix}
\eta_{s_{k-1}} \oplus (\nu_{s_{k-1}}\Delta t + \frac{1}{2} w_{s_k} \Delta t^2) \\
\nu_{s_{k-1}}+K(\nu_{d}-\nu_{s_{k-1}}) + w_{s_k} \Delta t
\end{bmatrix} \quad;\quad K=diag(k_1,k_2,k_3) \quad k_i>0\\\end{split}
\end{equation}
\sphinxAtStartPar
Where \(\eta_{s_{k-1}}\) is the previous 3 DOF robot pose (x,y,yaw) and \(\nu_{s_{k-1}}\) is the previous robot velocity (velocity in the direction of x and y B\sphinxhyphen{}Frame axis of the robot and the angular velocity).
\(u_{s_k}\) is the input to the motion model contaning the desired robot velocity in the x direction (\(u_d\)) and the desired angular velocity around the z axis (\(r_d\)).
\(w_{s_k}\) is the motion model noise representing an acceleration perturbation in the robot axis. The \(w_{s_k}\) acceleration is the responsible for the slight velocity variation in the simulated robot motion.
\(K\) is a diagonal matrix containing the gains used to drive the simulated velocity towards the desired input velocity.

\sphinxAtStartPar
Finally, the class updates the object attributes \(xsk\), \(xsk\_1\) and  \(usk\) to made them available for plotting purposes.

\sphinxAtStartPar
\sphinxstylestrong{To be completed by the student}.
\begin{quote}\begin{description}
\sphinxlineitem{Parameters}\begin{itemize}
\item {} 
\sphinxAtStartPar
\sphinxstyleliteralstrong{\sphinxupquote{xsk\_1}} \textendash{} previous robot state \(x_{s_{k-1}}=\begin{bmatrix}\eta_{s_{k-1}}^T & \nu_{s_{k-1}}^T\end{bmatrix}^T\)

\item {} 
\sphinxAtStartPar
\sphinxstyleliteralstrong{\sphinxupquote{usk}} \textendash{} model input \(u_{s_k}=\nu_{d}=\begin{bmatrix} u_d& r_d\end{bmatrix}^T\)

\end{itemize}

\sphinxlineitem{Returns}
\sphinxAtStartPar
current robot state \(x_{s_k}\)

\end{description}\end{quote}

\end{fulllineitems}

\index{ReadEncoders() (DifferentialDriveSimulatedRobot.DifferentialDriveSimulatedRobot method)@\spxentry{ReadEncoders()}\spxextra{DifferentialDriveSimulatedRobot.DifferentialDriveSimulatedRobot method}}

\begin{fulllineitems}
\phantomsection\label{\detokenize{robot_simulation:DifferentialDriveSimulatedRobot.DifferentialDriveSimulatedRobot.ReadEncoders}}
\pysigstartsignatures
\pysiglinewithargsret{\sphinxbfcode{\sphinxupquote{ReadEncoders}}}{}{}
\pysigstopsignatures
\sphinxAtStartPar
Simulates the robot measurements of the left and right wheel encoders.

\sphinxAtStartPar
\sphinxstylestrong{To be completed by the student}.
\begin{quote}\begin{description}
\sphinxlineitem{Return zsk,Rsk}
\sphinxAtStartPar
\(zk=[\Delta n_L~ \Delta n_R]^T\) observation vector containing number of pulses read from the left and right wheel encoders during the last differential motion. \(R_{s_k}=diag(\sigma_L^2,\sigma_R^2)\) covariance matrix of the read pulses.

\end{description}\end{quote}

\end{fulllineitems}

\index{ReadCompass() (DifferentialDriveSimulatedRobot.DifferentialDriveSimulatedRobot method)@\spxentry{ReadCompass()}\spxextra{DifferentialDriveSimulatedRobot.DifferentialDriveSimulatedRobot method}}

\begin{fulllineitems}
\phantomsection\label{\detokenize{robot_simulation:DifferentialDriveSimulatedRobot.DifferentialDriveSimulatedRobot.ReadCompass}}
\pysigstartsignatures
\pysiglinewithargsret{\sphinxbfcode{\sphinxupquote{ReadCompass}}}{}{}
\pysigstopsignatures
\sphinxAtStartPar
Simulates the compass reading of the robot.
\begin{quote}\begin{description}
\sphinxlineitem{Returns}
\sphinxAtStartPar
yaw and the covariance of its noise \sphinxstyleemphasis{R\_yaw}

\end{description}\end{quote}

\end{fulllineitems}

\index{ReadCartesian2DFeature() (DifferentialDriveSimulatedRobot.DifferentialDriveSimulatedRobot method)@\spxentry{ReadCartesian2DFeature()}\spxextra{DifferentialDriveSimulatedRobot.DifferentialDriveSimulatedRobot method}}

\begin{fulllineitems}
\phantomsection\label{\detokenize{robot_simulation:DifferentialDriveSimulatedRobot.DifferentialDriveSimulatedRobot.ReadCartesian2DFeature}}
\pysigstartsignatures
\pysiglinewithargsret{\sphinxbfcode{\sphinxupquote{ReadCartesian2DFeature}}}{}{}
\pysigstopsignatures
\sphinxAtStartPar
Simulates the reading of 2D cartesian features. The features are placed in the map in cartesian coordinates.
\begin{quote}\begin{description}
\sphinxlineitem{Returns}
\sphinxAtStartPar
\begin{description}
\sphinxlineitem{zsk: {[}{[}x1 y1{]},…,{[}xn yn{]}{]}}
\sphinxAtStartPar
Cartesian position of the feature observations.

\sphinxlineitem{Rsk: block\_diag(R\_1,…,R\_n), where R\_i={[}{[}r\_xx r\_xy{]},{[}r\_xy r\_yy{]}{]} is the}
\sphinxAtStartPar
2x2 i\sphinxhyphen{}th feature observation covariance.
Covariance of the Cartesian feature observations. Note the features are uncorrelated among them. They are independent. However, the x and y coordinates of each feature are correlated.

\end{description}


\end{description}\end{quote}

\end{fulllineitems}

\index{PlotRobot() (DifferentialDriveSimulatedRobot.DifferentialDriveSimulatedRobot method)@\spxentry{PlotRobot()}\spxextra{DifferentialDriveSimulatedRobot.DifferentialDriveSimulatedRobot method}}

\begin{fulllineitems}
\phantomsection\label{\detokenize{robot_simulation:DifferentialDriveSimulatedRobot.DifferentialDriveSimulatedRobot.PlotRobot}}
\pysigstartsignatures
\pysiglinewithargsret{\sphinxbfcode{\sphinxupquote{PlotRobot}}}{}{}
\pysigstopsignatures
\sphinxAtStartPar
Updates the plot of the robot at the current pose

\end{fulllineitems}


\end{fulllineitems}


\sphinxstepscope


\section{Robot Localization}
\label{\detokenize{Localization_index:robot-localization}}\label{\detokenize{Localization_index::doc}}
\sphinxstepscope


\subsection{Robot Localization}
\label{\detokenize{Localization:robot-localization}}\label{\detokenize{Localization::doc}}
\begin{figure}[htbp]
\centering

\noindent\sphinxincludegraphics[scale=0.75]{{Localization}.png}
\end{figure}
\index{Localization (class in Localization)@\spxentry{Localization}\spxextra{class in Localization}}

\begin{fulllineitems}
\phantomsection\label{\detokenize{Localization:Localization.Localization}}
\pysigstartsignatures
\pysiglinewithargsret{\sphinxbfcode{\sphinxupquote{class\DUrole{w}{ }}}\sphinxcode{\sphinxupquote{Localization.}}\sphinxbfcode{\sphinxupquote{Localization}}}{\sphinxparam{\DUrole{n}{index}}\sphinxparamcomma \sphinxparam{\DUrole{n}{kSteps}}\sphinxparamcomma \sphinxparam{\DUrole{n}{robot}}\sphinxparamcomma \sphinxparam{\DUrole{n}{x0}}\sphinxparamcomma \sphinxparam{\DUrole{o}{*}\DUrole{n}{args}}}{}
\pysigstopsignatures
\sphinxAtStartPar
Bases: \sphinxcode{\sphinxupquote{object}}

\sphinxAtStartPar
Localization base class. Implements the localization algorithm.
\index{\_\_init\_\_() (Localization.Localization method)@\spxentry{\_\_init\_\_()}\spxextra{Localization.Localization method}}

\begin{fulllineitems}
\phantomsection\label{\detokenize{Localization:Localization.Localization.__init__}}
\pysigstartsignatures
\pysiglinewithargsret{\sphinxbfcode{\sphinxupquote{\_\_init\_\_}}}{\sphinxparam{\DUrole{n}{index}}\sphinxparamcomma \sphinxparam{\DUrole{n}{kSteps}}\sphinxparamcomma \sphinxparam{\DUrole{n}{robot}}\sphinxparamcomma \sphinxparam{\DUrole{n}{x0}}\sphinxparamcomma \sphinxparam{\DUrole{o}{*}\DUrole{n}{args}}}{}
\pysigstopsignatures
\sphinxAtStartPar
Constructor of the DRLocalization class.
\begin{quote}\begin{description}
\sphinxlineitem{Parameters}\begin{itemize}
\item {} 
\sphinxAtStartPar
\sphinxstyleliteralstrong{\sphinxupquote{index}} \textendash{} Logging index structure (\sphinxcode{\sphinxupquote{prpy.Index}})

\item {} 
\sphinxAtStartPar
\sphinxstyleliteralstrong{\sphinxupquote{kSteps}} \textendash{} Number of time steps to simulate

\item {} 
\sphinxAtStartPar
\sphinxstyleliteralstrong{\sphinxupquote{robot}} \textendash{} Simulation robot object (\sphinxcode{\sphinxupquote{prpy.Robot}})

\item {} 
\sphinxAtStartPar
\sphinxstyleliteralstrong{\sphinxupquote{args}} \textendash{} Rest of arguments to be passed to the parent constructor

\item {} 
\sphinxAtStartPar
\sphinxstyleliteralstrong{\sphinxupquote{x0}} \textendash{} Initial Robot pose in the N\sphinxhyphen{}Frame

\end{itemize}

\end{description}\end{quote}

\end{fulllineitems}

\index{GetInput() (Localization.Localization method)@\spxentry{GetInput()}\spxextra{Localization.Localization method}}

\begin{fulllineitems}
\phantomsection\label{\detokenize{Localization:Localization.Localization.GetInput}}
\pysigstartsignatures
\pysiglinewithargsret{\sphinxbfcode{\sphinxupquote{GetInput}}}{}{}
\pysigstopsignatures
\sphinxAtStartPar
Gets the input from the robot. To be overidden by the child class.
\begin{quote}\begin{description}
\sphinxlineitem{Return uk}
\sphinxAtStartPar
input variable

\end{description}\end{quote}

\end{fulllineitems}

\index{Localize() (Localization.Localization method)@\spxentry{Localize()}\spxextra{Localization.Localization method}}

\begin{fulllineitems}
\phantomsection\label{\detokenize{Localization:Localization.Localization.Localize}}
\pysigstartsignatures
\pysiglinewithargsret{\sphinxbfcode{\sphinxupquote{Localize}}}{\sphinxparam{\DUrole{n}{xk\_1}}\sphinxparamcomma \sphinxparam{\DUrole{n}{uk}}}{}
\pysigstopsignatures
\sphinxAtStartPar
Single Localization iteration invoked from \sphinxcode{\sphinxupquote{prpy.DRLocalization.Localization()}}. Given the previous robot pose, the function reads the inout and computes the current pose.
\begin{quote}\begin{description}
\sphinxlineitem{Parameters}
\sphinxAtStartPar
\sphinxstyleliteralstrong{\sphinxupquote{xk\_1}} \textendash{} previous robot pose

\sphinxlineitem{Return xk}
\sphinxAtStartPar
current robot pose

\end{description}\end{quote}

\end{fulllineitems}

\index{LocalizationLoop() (Localization.Localization method)@\spxentry{LocalizationLoop()}\spxextra{Localization.Localization method}}

\begin{fulllineitems}
\phantomsection\label{\detokenize{Localization:Localization.Localization.LocalizationLoop}}
\pysigstartsignatures
\pysiglinewithargsret{\sphinxbfcode{\sphinxupquote{LocalizationLoop}}}{\sphinxparam{\DUrole{n}{x0}}\sphinxparamcomma \sphinxparam{\DUrole{n}{usk}}}{}
\pysigstopsignatures
\sphinxAtStartPar
Given an initial robot pose \(x_0\) and the input to the \sphinxcode{\sphinxupquote{prpy.SimulatedRobot}} this method calls iteratively \sphinxcode{\sphinxupquote{prpy.DRLocalization.Localize()}} for k steps, solving the robot localization problem.
\begin{quote}\begin{description}
\sphinxlineitem{Parameters}
\sphinxAtStartPar
\sphinxstyleliteralstrong{\sphinxupquote{x0}} \textendash{} initial robot pose

\end{description}\end{quote}

\end{fulllineitems}

\index{Log() (Localization.Localization method)@\spxentry{Log()}\spxextra{Localization.Localization method}}

\begin{fulllineitems}
\phantomsection\label{\detokenize{Localization:Localization.Localization.Log}}
\pysigstartsignatures
\pysiglinewithargsret{\sphinxbfcode{\sphinxupquote{Log}}}{\sphinxparam{\DUrole{n}{xsk}}\sphinxparamcomma \sphinxparam{\DUrole{n}{xk}}}{}
\pysigstopsignatures
\sphinxAtStartPar
Logs the results for later plotting.
\begin{quote}\begin{description}
\sphinxlineitem{Parameters}\begin{itemize}
\item {} 
\sphinxAtStartPar
\sphinxstyleliteralstrong{\sphinxupquote{xsk}} \textendash{} ground truth robot pose from the simulation

\item {} 
\sphinxAtStartPar
\sphinxstyleliteralstrong{\sphinxupquote{xk}} \textendash{} estimated robot pose

\end{itemize}

\end{description}\end{quote}

\end{fulllineitems}

\index{PlotXY() (Localization.Localization method)@\spxentry{PlotXY()}\spxextra{Localization.Localization method}}

\begin{fulllineitems}
\phantomsection\label{\detokenize{Localization:Localization.Localization.PlotXY}}
\pysigstartsignatures
\pysiglinewithargsret{\sphinxbfcode{\sphinxupquote{PlotXY}}}{}{}
\pysigstopsignatures
\sphinxAtStartPar
Plots, in a new figure, the ground truth (orange) and estimated (blue) trajectory of the robot at the end of the Localization Loop.

\end{fulllineitems}

\index{PlotTrajectory() (Localization.Localization method)@\spxentry{PlotTrajectory()}\spxextra{Localization.Localization method}}

\begin{fulllineitems}
\phantomsection\label{\detokenize{Localization:Localization.Localization.PlotTrajectory}}
\pysigstartsignatures
\pysiglinewithargsret{\sphinxbfcode{\sphinxupquote{PlotTrajectory}}}{}{}
\pysigstopsignatures
\sphinxAtStartPar
Plots the estimated trajectory (blue) of the robot during the localization process.

\end{fulllineitems}


\end{fulllineitems}



\subsection{Dead Reckoning}
\label{\detokenize{Localization_index:dead-reckoning}}
\sphinxstepscope


\subsubsection{3 DOF Differential Drive Mobile Robot Example}
\label{\detokenize{DRLocalization:dof-differential-drive-mobile-robot-example}}\label{\detokenize{DRLocalization::doc}}
\begin{figure}[htbp]
\centering

\noindent\sphinxincludegraphics[scale=0.75]{{DR_3DOFDifferentialDrive}.png}
\end{figure}
\index{DR\_3DOFDifferentialDrive (class in DR\_3DOFDifferentialDrive)@\spxentry{DR\_3DOFDifferentialDrive}\spxextra{class in DR\_3DOFDifferentialDrive}}

\begin{fulllineitems}
\phantomsection\label{\detokenize{DRLocalization:DR_3DOFDifferentialDrive.DR_3DOFDifferentialDrive}}
\pysigstartsignatures
\pysiglinewithargsret{\sphinxbfcode{\sphinxupquote{class\DUrole{w}{ }}}\sphinxcode{\sphinxupquote{DR\_3DOFDifferentialDrive.}}\sphinxbfcode{\sphinxupquote{DR\_3DOFDifferentialDrive}}}{\sphinxparam{\DUrole{n}{index}}\sphinxparamcomma \sphinxparam{\DUrole{n}{kSteps}}\sphinxparamcomma \sphinxparam{\DUrole{n}{robot}}\sphinxparamcomma \sphinxparam{\DUrole{n}{x0}}\sphinxparamcomma \sphinxparam{\DUrole{o}{*}\DUrole{n}{args}}}{}
\pysigstopsignatures
\sphinxAtStartPar
Bases: {\hyperref[\detokenize{Localization:Localization.Localization}]{\sphinxcrossref{\sphinxcode{\sphinxupquote{Localization}}}}}

\sphinxAtStartPar
Dead Reckoning Localization for a Differential Drive Mobile Robot.
\index{\_\_init\_\_() (DR\_3DOFDifferentialDrive.DR\_3DOFDifferentialDrive method)@\spxentry{\_\_init\_\_()}\spxextra{DR\_3DOFDifferentialDrive.DR\_3DOFDifferentialDrive method}}

\begin{fulllineitems}
\phantomsection\label{\detokenize{DRLocalization:DR_3DOFDifferentialDrive.DR_3DOFDifferentialDrive.__init__}}
\pysigstartsignatures
\pysiglinewithargsret{\sphinxbfcode{\sphinxupquote{\_\_init\_\_}}}{\sphinxparam{\DUrole{n}{index}}\sphinxparamcomma \sphinxparam{\DUrole{n}{kSteps}}\sphinxparamcomma \sphinxparam{\DUrole{n}{robot}}\sphinxparamcomma \sphinxparam{\DUrole{n}{x0}}\sphinxparamcomma \sphinxparam{\DUrole{o}{*}\DUrole{n}{args}}}{}
\pysigstopsignatures
\sphinxAtStartPar
Constructor of the \sphinxcode{\sphinxupquote{prlab.DR\_3DOFDifferentialDrive}} class.
\begin{quote}\begin{description}
\sphinxlineitem{Parameters}
\sphinxAtStartPar
\sphinxstyleliteralstrong{\sphinxupquote{args}} \textendash{} Rest of arguments to be passed to the parent constructor

\end{description}\end{quote}

\end{fulllineitems}

\index{Localize() (DR\_3DOFDifferentialDrive.DR\_3DOFDifferentialDrive method)@\spxentry{Localize()}\spxextra{DR\_3DOFDifferentialDrive.DR\_3DOFDifferentialDrive method}}

\begin{fulllineitems}
\phantomsection\label{\detokenize{DRLocalization:DR_3DOFDifferentialDrive.DR_3DOFDifferentialDrive.Localize}}
\pysigstartsignatures
\pysiglinewithargsret{\sphinxbfcode{\sphinxupquote{Localize}}}{\sphinxparam{\DUrole{n}{xk\_1}}\sphinxparamcomma \sphinxparam{\DUrole{n}{uk}}}{}
\pysigstopsignatures
\sphinxAtStartPar
Motion model for the 3DOF (\(x_k=[x_{k}~y_{k}~\psi_{k}]^T\)) Differential Drive Mobile robot using as input the readings of the wheel encoders (\(u_k=[n_L~n_R]^T\)).
\begin{quote}\begin{description}
\sphinxlineitem{Parameters}\begin{itemize}
\item {} 
\sphinxAtStartPar
\sphinxstyleliteralstrong{\sphinxupquote{xk\_1}} \textendash{} previous robot pose estimate (\(x_{k-1}=[x_{k-1}~y_{k-1}~\psi_{k-1}]^T\))

\item {} 
\sphinxAtStartPar
\sphinxstyleliteralstrong{\sphinxupquote{uk}} \textendash{} input vector (\(u_k=[u_{k}~v_{k}~r_{k}]^T\))

\end{itemize}

\sphinxlineitem{Return xk}
\sphinxAtStartPar
current robot pose estimate (\(x_k=[x_{k}~y_{k}~\psi_{k}]^T\))

\end{description}\end{quote}

\end{fulllineitems}

\index{GetInput() (DR\_3DOFDifferentialDrive.DR\_3DOFDifferentialDrive method)@\spxentry{GetInput()}\spxextra{DR\_3DOFDifferentialDrive.DR\_3DOFDifferentialDrive method}}

\begin{fulllineitems}
\phantomsection\label{\detokenize{DRLocalization:DR_3DOFDifferentialDrive.DR_3DOFDifferentialDrive.GetInput}}
\pysigstartsignatures
\pysiglinewithargsret{\sphinxbfcode{\sphinxupquote{GetInput}}}{}{}
\pysigstopsignatures
\sphinxAtStartPar
Get the input for the motion model. In this case, the input is the readings from both wheel encoders.
\begin{quote}\begin{description}
\sphinxlineitem{Returns}
\sphinxAtStartPar
uk:  input vector (\(u_k=[n_L~n_R]^T\))

\end{description}\end{quote}

\end{fulllineitems}


\end{fulllineitems}



\begin{savenotes}\sphinxattablestart
\sphinxthistablewithglobalstyle
\sphinxthistablewithnovlinesstyle
\centering
\begin{tabulary}{\linewidth}[t]{\X{1}{2}\X{1}{2}}
\sphinxtoprule
\sphinxtableatstartofbodyhook\sphinxbottomrule
\end{tabulary}
\sphinxtableafterendhook\par
\sphinxattableend\end{savenotes}


\chapter{Indices and tables}
\label{\detokenize{index:indices-and-tables}}\begin{itemize}
\item {} 
\sphinxAtStartPar
\DUrole{xref,std,std-ref}{genindex}

\item {} 
\sphinxAtStartPar
\DUrole{xref,std,std-ref}{modindex}

\item {} 
\sphinxAtStartPar
\DUrole{xref,std,std-ref}{search}

\end{itemize}



\renewcommand{\indexname}{Index}
\printindex
\end{document}